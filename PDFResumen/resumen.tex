\documentclass[10pt,a4paper,spanish]{article}

\usepackage[spanish]{babel}
\usepackage[utf8]{inputenc}
\usepackage{amsmath, amsthm}
\usepackage{amsfonts, amssymb, latexsym}
\usepackage{enumerate}
% \usepackage[official]{eurosym}
\usepackage{graphicx}
\usepackage[usenames, dvipsnames]{color}
\usepackage{colortbl}
\usepackage{multirow}
\usepackage{fancyhdr}
\usepackage[all]{xy}
% \usepackage{minted}
\usepackage{float}
\usepackage{subfigure}
\usepackage{tikz}
\usepackage{pgfplots}
\usepackage{cancel}
\pgfplotsset{compat=1.5}

\usepackage[top=2.5cm, bottom=2.5cm, left=3cm, right=3cm]{geometry}

\usepackage[bookmarks=true,
            bookmarksnumbered=false, % true means bookmarks in
                                     % left window are numbered
            bookmarksopen=false,     % true means only level 1
                                     % are displayed.
            colorlinks=true,
            linkcolor=red,
            citecolor=blue]{hyperref}

\newcommand{\HRule}{\rule{\linewidth}{0.5mm}} % regla horizontal para  el titulo

\pagestyle{plain}
%con esto nos aseguramos de que las cabeceras de capítulo y de sección vayan en minúsculas

\fancyhf{} %borra cabecera y pie actuales
% \fancyhead[LE,RO]{}
% \fancyhead[LO]{}
\fancyfoot[C]{\thepage}
% \renewcommand{\headrulewidth}{0.5pt}
% \renewcommand{\footrulewidth}{0pt}
% \addtolength{\headheight}{0.5pt} %espacio para la raya
% \fancypagestyle{plain}{%
%       \fancyhead{} %elimina cabeceras en páginas "plain"
%       \renewcommand{\headrulewidth}{0pt} %así como la raya
% }

% %%%%% Para cambiar el tipo de letra en el título de la sección %%%%%%%%%%%
% \usepackage{sectsty}
% \chapterfont{\fontfamily{frc}\selectfont}
% \sectionfont{\fontfamily{pag}\selectfont}
% \subsectionfont{\fontfamily{pag}\selectfont}
% \subsubsectionfont{\fontfamily{pag}\selectfont}

% \newmintedfile[mycplusplus]{c++}{
%     linenos,
%     numbersep=5pt,
%     gobble=0,
%     frame=lines,
%     framesep=2mm,
%     tabsize=3,
% }

% \newmintedfile[mypython]{python}{
%     linenos,
%     numbersep=5pt,
%     gobble=0,
%     frame=lines,
%     framesep=2mm,
%     tabsize=3,
% }

\definecolor{amaranth}{rgb}{0.9, 0.17, 0.31}

\usepackage{arev}
\usepackage[T1]{fontenc}

\setlength{\parindent}{0pt}
\setlength{\parskip}{1ex plus 0.5ex minus 0.2ex}

% \usepackage{titlesec}

% % \titleformat{\chapter}{\normalfont\huge\center}{--- \thechapter ---}{20pt}{}

% \titleformat
% {\chapter} % command
% [display] % shape
% {\Huge\center\bfseries} % format
% {--- \thechapter ---} % label
% {0.5ex} % sep
% {
%     \rule{\textwidth}{1pt}
%     \vspace{1ex}
%     \centering
% } % before-code
% [
% \vspace{-0.5ex}%
% \rule{\textwidth}{0.3pt}
% ] % after-code

%Definimos autor y título
\title{\Huge Arquitectura de las Redes \textit{Tor}}
\author{\Large Marta Gómez Macías y Braulio Vargas López}

\begin{document}
\renewcommand{\tablename}{Tabla}
\maketitle

\section{¿Qué es \textit{Tor}?}
Según \cite{deftor}, ``la red Tor es un grupo de servidores operativos voluntarios que permiten a las personas mejorar su privacidad y seguridad en Internet''. Esto quiere decir que una conexión a través de Tor nunca será directa, sino que pasará por estos servidores voluntarios para que no se pueda saber quién manda el paquete ni a dónde va dirigido. Además, Tor también se define como ``una efectiva herramienta para la elusión de la censura, permitiendo a sus usuarios acceder a contenido que de otra forma se encontraría bloqueado.''

\subsection{¿Por qué es \textit{Tor} más seguro que otras herramientas?}
Tal y como se explica en \cite{deftor}, usando Tor nos protegemos contra el ``análisis de tráfico''. Aunque el \textit{payload} de los paquetes que enviamos a través de la red se encuentre encriptado, la cabecera normalmente no suele estarlo ya que se necesita para dirigir el paquete. Ésta cabecera facilita a los ``sniffers'' muchísima información sobre lo que estamos haciendo ya que incluye información como el ``host'' emisor, el ``host'' destino, el tamaño, el puerto al que va dirigido, etc.

¿La solución? Usar una red distribuida y anónima.

\section{¿Cómo mantiene \textit{Tor} el anonimato?}

El anonimato en \textit{Tor} se mantiene distribuyendo los paquetes que enviamos y recibimos por una gran cantidad de puntos distribuidos por internet, antes de llegar al destino, en vez de hacer una conexión directa entre el origen y el destino. Los paquetes de datos seguirán un camino aleatorio entre los distintos ``relevos'' que existen en la red de Tor, que eliminan nuestras $huellas$, para que en ningún punto del camino que toma el paquete se pueda saber de dónde viene el paquete y a dónde va.

Para crear un camino en la red privada de Tor, se crea un circuito incremental de conexiones encriptadas a través de los ``relevos'' de la red. Los incrementos del circuito se hacen de uno en uno cada vez, y cada punto del circuito sólo sabe de qué relevo viene el paquete, y a qué relevo se lo tiene que dar. El cliente aporta un conjunto de llaves diferentes para encriptar el mensaje para cada punto del circuito para asegurarse de que en cada punto del circuito no se pueda seguir el paquete.

\section{Arquitectura}

\section{Protocolo}

\section{Ejemplo con Wireshark}

\bibliography{resumen.bib} %archivo citas.bib que contiene las entradas 
\bibliographystyle{siam} % haycle varias formas de citar

\end{document}